\documentclass{jsarticle}
\usepackage[dvipdfmx]{graphicx}
\title{ウェルネス実習 S2木(申告専用) レポート}
\renewcommand{\abstractname}{課題}
\author{******** ****}
\date{}
\begin{document}
\maketitle

\begin{abstract}
授業中にバスケットボールのゲームを実施するためにどのような準備をしているか。授業に参加したことがない人にも分かるように示しなさい。安全をキーワードに、その理由も述べること。
\end{abstract}

\section{コートのモップ掛け}

授業が始まる前に、バスケットボールコート内をモップ掛けする。モップ置き場は、体育館に入って左手奥にある。砂塵などで足が滑るのを防ぐ目的がある。

\section{用具の準備}

必要な用具は、バスケットボール、デジタルタイマー、ホワイトボード、ビブス、電子ホイッスル、雑巾である。ボール、タイマー、ホワイトボードは、体育館に入って左手前の器具庫内にある。その他のものは、かごに入った状態で、体育館地下の事務室内にある。雑巾は、シューズが滑るのを防ぐためにある。

\section{ゲーム中の役職}

審判とタイマー係を、ゲームに参加しないチームのメンバーで請け負う。審判は電子ホイッスルを持ち、得点やルールに反した行為などをチェックする。危険なプレイをさせないようにする目的がある。タイマー係は、ゲームの開始とともにタイマーを6分でスタートし、得点が入ったときにカウンターの数字を増やす。

\end{document}
