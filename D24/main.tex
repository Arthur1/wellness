\documentclass{jsarticle}
\usepackage[dvipdfmx]{graphicx}
\title{生涯スポーツ実習D24 レポート\\「3x3について」}

\author{******** ****}
\begin{document}
\date{}
\maketitle

\section{概要}

バスケットボールの「3x3」は、街中のコートで行われていた「3on3」というハーフコートの3人制ゲームに、国際バスケットボール連盟(FIBA)が世界統一のルールを設け、種目化したものである。2007年に種目として確立し、2017年には2020東京オリンピックの追加種目に採択された。

\section{ルール}

3x3では、通常の5人制ゲームとは異なるルールが採用されている。得点はアーク内ならば1点、外ならば2点となる。1試合の時間は10分であるが、21点先取した時点でそのチームの勝利となる(ノックアウト制)。ボールを持ってから12秒以内に攻め切らなければならない。

\section{文化的背景}

ストリートスポーツから発展して種目化した経緯からか、大会もストリート文化を意識したカジュアルな形式で行われることが多い。日本の3x3トップリーグ「3x3 PREMIER.EXE」では、街中の商業施設などにコート・ゴールを設置して行う大会を開催している。DJブースが設置されヒップホップ・ミュージックが流れたり、MCが選手を囃し立てたりと、エンターテインメント性も重視されている。

\begin{thebibliography}{1}
  \bibitem{1} 「JBA 3x3 Official Web Site/公益財団法人日本バスケットボール協会(JBA)」http://3x3.japanbasketball.jp/ (2018/1/4アクセス)
  \bibitem{2} 「3x3.EXE | 3人制バスケットボールリーグ」http://3x3exe.com/ (2018/1/4アクセス)
\end{thebibliography}

\end{document}
