\documentclass{jsarticle}
\usepackage[dvipdfmx]{graphicx}
\title{生涯スポーツ実習D21 レポート\\「2017年・2018年の卓球ニュース」}

\author{******** ****}
\date{}
\begin{document}
\maketitle

\section{世界卓球2017ドイツ}

2017年5月29日から6月5日の間、「世界卓球2017ドイツ」が開催された。日本代表は混合ダブルスの吉村真晴・石川佳純ペアが48年ぶりの金メダルを獲得した。他にも、女子シングルスでは平野美宇が銅メダル(48年ぶり)、男子ダブルスでは大島祐哉・森薗政崇ペアの銀メダル(48年ぶり)に加え吉村真晴・丹羽孝希ペアの銅メダル、女子ダブルスでは伊藤美誠・早田ひなペアが銅メダル(16年ぶり)を獲得した。男子シングルスにて13歳の張本智和が史上最年少ベスト8入りしたことも話題となった。

\section{平成29年度全日本卓球選手権大会}

2018年1月15日から21日の間、「天皇杯・皇后杯 平成29年度全日本卓球選手権大会(ジュニア・一般の部)」が行われる(現在開催中)。1月18日に行われたジュニア男子シングルス決勝にて、張本智和が大会初優勝した。張本は一般の部のシングルス・ダブルスにも出場する。同日の混合ダブルス決勝では、森薗政崇・伊藤美誠ペアが大会初優勝した。世界卓球2017にて金メダルを獲得した吉村真晴・石川佳純ペアは準決勝で敗退した。

\begin{thebibliography}{1}
  \bibitem{1} 「日本代表5つのメダル獲得 世界卓球2017ドイツ 結果一覧|テレビ東京卓球NEWS:テレビ東京」http://www.tv-tokyo.co.jp/tabletennis/news/2017/06/001695.html (2018/1/19アクセス)
  \bibitem{2} 「森薗、伊藤組が初優勝=ジュニアで張本が最年少V-全日本卓球:時事ドットコム」https://www.jiji.com/jc/article?k=2018011800841\&g=spo (2018/1/19アクセス)
\end{thebibliography}

\end{document}
